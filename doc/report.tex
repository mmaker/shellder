\documentclass[11pt]{article}
\usepackage{enumitem}
\usepackage{hyperref}
\title{\textbf{Sistemi Operativi 1}}
\author{Michele Orr\`u}
\date{}
\begin{document}

\newcommand{\code}{\texttt}

\maketitle

\section{Introduzione}

\textbf{Shellder} \`e una \emph{shell interattiva} con \emph{scheduling prioritario}.
Ogni comando digitato dall'utente viene posto in una coda prioritaria ed
eseguito in background, secondo delle particolari dinamiche di scheduling. Lo
standard output di ogni processo viene visualizzato all'interno di un apposito
dile di log \code{/tmp/shellder.log},  se non diversamente indicato da
\emph{command-line}.

Lo scopo d iquesto documento e` quello di fornire un'inquadratura generale di
come \`e stato organizzato il codice, delle soluzioni algoritmiche trovate ai
vari problemi, e di come il tutto \`stato integrato con le \emph{system-call}
offerte dal sistema operativo. Tutti i dettagi implementativi sono stati
delegati alla documentazione automaticamente generata con
\href{http://www.stack.nl/~dimitri/doxygen/}{Doxygen}.

\section{Utilizzo del software}
\label{usage}

Prima di cominciare, \`e necessario installare gli headers della libreria libreadline,
nel caso della debian wheezy presa in considerazione questo e` possibile tramite:
\\ \code{\$ sudo apt-get install libreadline-dev}

\noindent Spostarsi all'interno della root del progetto, quindi compilare il codice con:
\\ \code{\$ make}

\noindent \`E ora possibile eseguire il software:
\\ \code{\$ ./shellder -h}

\noindent Verr\`a visualizzato l'help del programma, e listate le opzioni mediante le quali e`
possibile modificare le costanti usate dallo schedule e cambiare la destinazione del file di log.
Alcuni esempi di come e` possibile usare la shell potrebbero essere:
\\ \code{\$ ./shellder  -T 2  -B -0.1  -A 1  -l shellder.log}
\\ \code{>>> ./examples/ping.py }
\\ \code{>>> ./examples/ping.py }
\\ \code{>>> ./examples/ping.py }

\noindent Nella prima riga viene invocata la shell; viene quindi riempita la
coda dei processi che gestisce lo scheduler.
\\ \code{>>> \% ps}

\noindent Il seguente \emph{comando di debug} mostrer\`a la lista dei processi
attualmente in coda.
\\ \code{>>>  \textasciicircum D}

\noindent Premere \code{Ctrl-D} per uscire.

\section{Albero delle directory}

L'intero progetto presente in allegato e` stato sviluppatro tramite git ed
\`pubblicamente accessibile su
\href{https://bitbucket.org/maker/shellder/}{Bitbucket}

La struttura del filesystem segue pressoche` quella di un qualsiasi altro progetto \code{C}:
la root del sorgente si chiama \code{shellder/}, e all'interno di essa \`e possibile trovare i
file \code{LICENSE}, \code{README}.

Vi \`e infine la cartella \code{src/}, contentente il sorgente dell'applicazione:
\begin{itemize}
\item \code{shell.[ch]}, che ha il compito di gestire l'interazione con l'utente, la command line,
       e l'inserimento di nuovi processi nella coda dello scheduler;

\item \code{process.[ch]} nella quale viene definito il tipo \texttt{proc\_t}, utilizzato per memorizzare
	   tutte le informazioni utili di un processo in esecuzione, e le funzioni per interagire con esso;

\item \code{scheduler.[ch]}, nel quale viene definita la coda dei processi, le operazioni da
       effettuare su essa, e le funzioni di inizializzazione del thread dedicato allo scheduling;

\item \code{utils.[ch]}, nel quale sono implementate diverse operazioni per la manipolazione di
       stringhe e/o di strutture usate internamente al programma;

\item \code{dbg.[ch]} contenente tutte le funzioni di debug.
\end{itemize}

\section{Scelte Implementative}

\paragraph{Logging\\}
Il file di log di default \`e stato settato di default a un path diverso da
quello richiesto nella consegna semplicemente per esigenze di sviluppo: la
scrittura dentro la directory \code{/var/log} avrebbe infatti disturbato il
monitoraggio dei log di sistema, e perch\`e sui moderni computer dotati di
\emph{SSD} \`e indicato montare la directory \code{/tmp} in RAM, risparmiando
cicli di scrittura potenzialmente erosivi, specialmente in caso di bug.

Per gli stessi motivi, l'iniziale scelta di adottare il log di sistema
\code{syslog} \`e stata abbandonata - l'implementazione \`e ancora presente, ma
commentata.

\paragraph{Ordinamento dei Processi\\}
L'algoritmo dedito allo scheduling segue palesemente una coda prioritaria;
Per esigenze didattiche abbiamo deciso di non appoggiarci a progetti di terze
parti, fuori dalla libreria standard, quanto piuttosto implementarne una nostra versione,
facendo riferimento all'algoritmo proposto da \emph{Introduction to Algorithms, The MIT Press}.
La sua implementazione \`e consultabile all'interno del file \code{scheduler.c}.

\paragraph{Tools per il debugging\\}
Al di l\`a dell'ausilio software esterni quali \code{gdb} e \code{valgrind},
all'interno della shell stessa \`e stato implementato un meccanismo per aiutare
nel processo di debugging il controllo della correttezza dello scheduler.
Infatti, tutti i comandi il cui primo carattere \`e "\code{\%}" vengono processati
al di fuori dalla coda dei processi, da delle funzioni interne alla shell.
\`E possibile listare le funzioni di debug mediante il comando \code{\%help}.

Inoltre, per testare lo scheduler, sono disponibili alcuni eseguibili all'interno della
cartella \code{examples/}: trattasi di piccoli script che interagiscono con lo standard
output e che possono essere usati per valutare, dall'esterno, le performance dell'applicazione.
Ad esempio, lo script \code{ping.py} mostrato in \nameref{usage} consta di un
loop infinito che scrive a intervalli regolari il proprio pid.

\section{Valutazione delle Costanti}

La priorit\`a di un processo viene valutata da una funzione riportata nella consegna:
$$
P(t) = f(t) = a \cdot e^{-bt}
$$
Dove la variabile $t$, non essendo stato esplicitamente indicato, e` il tempo totale
 trascorso dall'esecuzione della shell.

Le costanti $a$ e $b$  invece possono essere determinate dell'utente per gestire la
modalita` secondo cui effettuare lo scheduling. Analizziamo di seguito come la funzione
$f(t)$ degenera per valori particolari di queste:
\begin{itemize}
\item Per valori di $b$ che tendono a $0$ l'algoritmo degenera in un \emph{round-robin}, ossia ai processi vengono assegnati eguali \emph{time-slices}, senza alcuna forma di prioritizzazione:
$$
\forall \ a \quad \lim_{b \to 0^+}{f(t)} = a
$$

\item $if\ b\ != 0$, $a$ \`e trascurabile, in quanto fattore moltiplicativo; tuttavia, per valori
molto grandi di $a$, si potrebbero avere casi di \emph{integer overflow}, se l'informazione
necessaria per rappresentare $f(t) > $ \code{sizeof(long double)};

\item per tutti gli altri casi, la variabile $t$ gioca un riolo fondamentale: \emph{decrementa} la priorit\`a
     del processo finch\`e questo non converge a $0$:
$$
\lim_{t \to +\infty}f(t) = a \cdot e^{-bt} = a \cdot e^{-t} = \frac{a}{e^{-t}} = 0
$$

Da un punto di vista puramente pratico, \`e possibile apprezzare bene questo fenomeno per valori
$a \in [] \quad b \in []$.
Ad esempio
\\ \code{>>> \%set }

\end{itemize}


\section{Corner Cases e limiti applicativi}

Attualmente l'applicazione soffre di tutta una serie di problemi di sicurezza,
sia dettati dal tempo, sia dal design stesso dell'applicazione. Di seguito sono
riportati quelli pi\`u rilevanti, e le modalit\`a in cui potrebbero essere
risolti.

\begin{itemize}
\item gli input da command line non vegnono gestiti, specialmente nel caso in cui si desidera convertire una stringa in numero.
      Questo potrebbe comportare \emph{errori di segmentazione} al runtime, quando invece l'output desiderato sarebbe la stampa
      delle opzioni disponibili  in combinazione con \texttt{EXIT\_FAILURE}.

\item i processi che vengono messi in coda, sono gestiti dallo scheduler mediante
      comunicazione \emph{IPC}, usando i segnali
      \code{SIGSTOP} e \code{SIGCONT}. Tuttavia, lo standard posix non
      implementa la system call \code{kill} per gruppi di processi. Questo vuol
      dire che un potenziale programma che fa uso della \code{fork()} potrebbe
      compromettere completamente l'algoritmo di scheduling;

\item la shell pi\`u usata in ambito unix, \code{bash}, e` strutturata di modo
      che il segnale \code{SIGINT} viene inoltrato a tutti i processi figli.
      Questo implica che se viene premuto \code{Ctrl-C} il segnale verra` ricevuto
      anche dai processi figli;

\item l'intera coda dei prcessi viene gestita tramite un vettore di grandezza \code{HEAPLEN}
      staticamente definito. Questo comporta un limite nel numero di processi potenzialmente
      gestibili in background. Possibili soluzioni sarebbero l'implementazione di una chiamata alla
      system-call \code{resize()} nel metodo \code{pinsert()} per una riallocazione
      dell'intera coda, mappandola su un vettore piu` grande.
\end{itemize}
\end{document}
