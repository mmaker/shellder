\documentclass[11pt]{article}
\usepackage{enumitem}
\title{\textbf{Sistemi Operativi 1}}
\author{Michele Orr\`u}
\date{}
\begin{document}

\maketitle

\section{Introduzione}

\textbf{Shellder} \`e una \emph{shell interattiva} con \emph{scheduling prioritario}.
Ogni comando digitato dall'utente viene posto in una coda prioritaria ed eseguito in background nella shell di
default \texttt{/bin/sh}. Questo vuol dire che ogni comando scritto sul \emph{Shellder} \`e equivalente ad un
comando sulla shell standard: variabili d'ambiente come \texttt{\$PATH} avranno lo stesso significato;
con il vantaggio di poter eseguire i vari processi un background e di poterli gestire attraverso uno 
\emph{scheduler spaziotemporale}.
L'output di questi processi, insieme ai log della shell interattiva, vengono salvati all'interno di un file 
\texttt{shellder.log}, se non diversamente indicato.

\section{Albero delle directory}

L'intero progetto pu\`o essere scaricato in qualsiasi momento da GitHub[0].
\texttt{git clone https://github.com/mmaker/shellder.git}

La struttura del filesystem segue pressoche` quella di un qualsiasi altro progetto C: la root del sorgente si chiama \texttt{shellder/}, e all'interno di essa \`e possibile trovare i file \texttt{LICENSE}, \texttt{README}.


\section{Scelte Implementative}

\paragraph{Ordinamento dei Processi\\}
L'algoritmo dedito allo scheduling segue palesemente una coda prioritaria;
le possibilit\`a di implementazione di quest'ultimo sono numerosissime e
spaziano dal vettore statico alle librerie esterne di ricerca scientifica.

Per esigenze didattiche abbiamo deciso di non appoggiarci a progetti di terze
parti non inclusi nella libreria standard c; dopo aver valutato mediante analisi
dei costi, abbiamo optato per una struttura della forma linked tree piuttosto
che un vettore statico, difatti costi di inserimento e rimozione risultano
ordini di grandezza minori nel caso dell'albero linkato:



\end{document}
